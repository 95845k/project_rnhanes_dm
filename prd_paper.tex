\documentclass[twoside,11pt]{article}

% Any additional packages needed should be included after jmlr2e.
% Note that jmlr2e.sty includes epsfig, amssymb, natbib and graphicx,
% and defines many common macros, such as 'proof' and 'example'.
%
% It also sets the bibliographystyle to plainnat; for more information on
% natbib citation styles, see the natbib documentation, a copy of which
% is archived at http://www.jmlr.org/format/natbib.pdf

\usepackage{jmlr2e}
%\usepackage{parskip}
\usepackage{enumitem}
\setlist{nolistsep,leftmargin=*}

% Definitions of handy macros can go here
\newcommand{\dataset}{{\cal D}}
\newcommand{\fracpartial}[2]{\frac{\partial #1}{\partial  #2}}
% Heading arguments are {volume}{year}{pages}{submitted}{published}{author-full-names}

% Short headings should be running head and authors last names
\ShortHeadings{95-845: MLHC Article}{Lastname and Lastname}
\firstpageno{1}

\begin{document}

\title{Heinz 95-845: \\Using Machine Learning to Predict Undiagnosed Diabetes \\in the NHANES Dataset}

\author{\name Kevin McGrady \email kevinmcg@andrew.cmu.edu \\
       \addr Heinz College\\
       Carnegie Mellon University\\
       Pittsburgh, PA, United States} 

\maketitle

\begin{abstract}
  The abstract is the summary of the article. Your potential readers will glance at the abstract to decide
  if the article is worth reading. Make it good--this is your most-read text!  
\end{abstract}

\section{Introduction}
The American Diabetes Association states that 29 million Americans have diabetes and of that, 8 million are undiagnosed. Diabetes ranks seventh in leading causes of death in the United States [1]. Detecting individuals that are likely to have undiagnosed diabetes and properly diagnosing them will help decrease the risk of diabetes-related complications like cardiovascular disease [2]. Application of machine learning algorithms can help determine which individuals should be screened. 

Yu et al. used support vector machines to classify individuals into diabetes and non-diabetes groupings [3]. Omogbai used support vector machines with canonical correlation analysis [4]. Dall et al. used logistic regression [5]. Jain used a feed forward neural network [6]. Barber et al. cite 18 models in their review of pre-diabetes risk assessment tools - 11 used logistic regression, 6 used decision trees, and 1 used support vector machines (Yu) [7].

In contrast, this analysis applies multiple machine learning methods to execute a classification task that is strictly focused on the prediction of undiagnosed diabetes. Yu and Omogbai sought to assign the correct label between (A) diagnosed or undiagnosed diabetes versus pre-diabetes or no diabetes and (B) undiagnosed diabetes or pre-diabetes versus no diabetes. Dall had a multi-class prediction, assigning no diabetes, pre-diabetes, and undiagnosed diabetes. Jain estimated individual level-risk of diabetes.

In Section 2, Background, information is provided on diabetes, machine learning, and NHANES. In Section 3, Model, information is provided on the selected machine learning algorithms. 

The layout of the paper is as follows:
Background (Section 2), Model (Section 3), Methods (Section 4), Results (Section 5), Discussion (Section 6), and \\Conclusion (Section 7). 

\section{Background} \label{background}
Diabetes is the common name for the metabolic disease, diabetes mellitus. The disease arises from problems within the relationship between the pancreas, insulin, glucose (sugar), the bloodstream, and the cells within the bloodstream. An improper balance of blood sugar can have many deleterious effects. There are three primary forms of diabetes: Type 1, Type 2, and gestational [8-10]. For this analysis, there is no differentiation made between Type 1 and Type 2 diabetes. Gestational diabetes is not included. 

Machine learning, in this application, entails solving a classification task. Computer algorithms are employed to predict whether to classify an individual as an undiagnosed diabetic or not. For this study, six different algorithms are applied. More information on the types of algorithms used in this study can be found in Section 3, Model. 

NHANES is the National Health and Nutrition Examination Survey issued by the National Center for Health Statistics, which is part of the Centers for Disease Control and Prevention [11]. The NHANES components selected for this study were questionnaire, laboratory, demographics, and examination. For more information on NHANES data subset used in this analysis, refer to Sections 4-4.3. 

\section{Model} \label{model}
This analysis applies logistic regression, decision tree, naive bayes, tree augmented naive bayes, support vector machines, and random forest algorithms to the task of predicting which individuals have undiagnosed diabetes. The entirety of the code is written in the R Project for Statistical Computing. The libraries used are as follows: stats, rpart, bnlearn, e1071, and randomForest. Documentation for these libraries can be found at \url{https://cran.r-project.org/web/packages}. All code for this project is available at \url{https://github.com/95845k/project_rnhanes_dm}.

\section{Methods} \label{methods}
The following subsections explain how the data was extracted (4.1), how the primary outcome was derived (4.2), how features were selected (4.3), how the data was transformed (4.4), and how the models were evaluated (4.5-4.6). 

\subsection{Data Extraction} 
Data for this project was extracted from NHANES using the R library RNHANES [12]. Selected files were downloaded from the period 1999-2000 through 2013-2014 (see: \url{https://github.com/95845k/project_rnhanes_dm/blob/master/prd_code_data_extract.R}. Survey response codes were re-coded to response meanings and checked (see: \url{https://github.com/95845k/project_rnhanes_dm/blob/master/prd_code_data_extract_checks.R}. After joining all of the datasets, 82,091 records were present. 

\subsection{Outcome Derivation} 
Following Yu, the study dataset was filtered individuals aged 20 and above who were not pregnant at the time of the survey [3]. The age filter eliminated 38,298 records and following that, the pregnancy filter removed 1,416. 

The primary outcome was determined by both questionnaire and laboratory results. If the response was anything other than "No" to the question, "Other than during pregnancy, have you ever been told by a doctor or health professional that you have diabetes?" then the record was removed. This question posed as the proxy for diabetes diagnosis. The study dataset included only those individuals that did not have a diabetes diagnosis. The result of this question removed 5,757 records. To determine whether an individual had diabetes three laboratory tests were employed: fasting plasma glucose (FPG), A1c, and oral glucose tolerance test (OGTT). Following Dall, diabetes was defined as having either FPG greater than or equal 126 or A1c greater than or equal to 6.5 or OGTT greater than or equal to 200 [5]. If an individual did not have any of the three lab values present, then that individual record was eliminated from the dataset. This removed 3,527 records. Those that answered "No" to diagnosis and had a lab value in those ranges were considered to be undiagnosed diabetics. The final study dataset contained 33,093 records. 

\subsection{Feature Choices} 
Describe the data. Provide information about the population, the inclusion and exclusion criteria, what data were extracted, how features were processed, etc.

What features were used? What conversions were necessary? What assumptions (e.g. i.i.d.) are made? with how you might have converted the raw data into features that were used in your algorithm. 

Describe how the samples you used were selected to form your cohort and also to provide cohort descriptive statistics. In methodologic papers, the ``Table 1'' describing the population by covariate summary statistics goes here. In application papers, ``Table 1'' leads the Results Section. Relevant information about the study design, such how cases and controls were identified, goes here. See Section \ref{results} for an example of how to build a table in LaTeX.


\subsection{Data Transformation} 
How did you deal with missing data? 

\subsection{Comparison Methods}
train/test
To evaluate your model, often times you will compare against existing models.
If so, include them here with a brief description, citation, and any tweaks you made for your experiment.

\subsection{Evaluation Criteria}
Evaluation methods belong here as well.
Perhaps you used accuracy and the AUROC--explain why these are most useful measures of the outcome.

\section{Results} \label{results}

Present the results here.
Do not describe how the results were obtained.
Those descriptions belong in Section \ref{experiment}.

Typically there are multiple parts and subparts of your study.
Use subsections to report the results.

\subsection{Results on Application A} 

Give us some numbers about how well your method works, especially in comparison to some baselines.
You should provide a summary of the results in the text, as well as in tables (such as table~\ref{tab:example}) and figures (such as figure~\ref{fig:example}).  

You may use subfigures/wrapfigures (LaTeX packages) so that figures don't have to span the whole page or multiple figures are side by side.

\begin{table}[htbp]
  \centering 
  \begin{tabular}{lclc} 
    Method & Outcome (\%) \\ 
    \hline \\[-11pt]
    Us & 20.1 \\ 
    Baseline & 18.2 \\ \hline 
  \end{tabular}
  \label{tab:example} 
    \caption{Outcome by method used. These are our results.} 
\end{table}

\begin{figure}[htbp]
  \centering 
  \includegraphics[width=1.5in]{smile.jpeg} 
  \caption{Example smile graphic.}
  \label{fig:example} 
\end{figure} 

\subsection{Results on Application B} 

Did more than one experiment type?

\section{Discussion and Related Work} 

This is where you characterize the outcomes of your method and draw conclusions from you experiment.
The discussion will build upon the Introduction and the Results sections to synthesize where your contribution brings the field. Discuss any implications of your work. 
Discuss limitations of your work.
Are there situations where you should and should not use your method.
What implications are there on policy making, clinical decision making, or future research activities?
Remember to contextualize your work with respect to related work and provide references.

\section{Conclusion} 
Summarize your work one more time, this time assuming the reader has read your paper.
Build suspense for what your next extension to this method would be.

% ACKNOWLEDGEMENTS ONLY GO IN THE CAMERA-READY, NOT THE SUBMISSION
% \acks{Many thanks to all collaborators and funders!}

\section{Bibliography}
\begin{enumerate}
\item http://www.diabetes.org/diabetes-basics/statistics
\item American Diabetes Association. "Standards of medical care in diabetes - 2013." Diabetes care 36. Supplement 1 (2013): S13.
\item Yu, Wei, et al. "Application of support vector machine modeling for prediction of common diseases: the case of diabetes and pre-diabetes." BMC Medical Informatics and Decision Making 10.1 (2010): 16.
\item Omogbai, Aileme. "Application of multiview techniques to NHANES dataset." arXiv preprint arXiv:1608.04783 (2016).
\item Dall, Timothy M., et al. "Detecting type 2 diabetes and prediabetes among asymptomatic adults in the United States: modeling American Diabetes Association versus US Preventive Services Task Force diabetes screening guidelines." Population health metrics 12.1 (2014): 12.
\item Jain, Deepti, and Divakar Singh. "A Neural Network based Approach for the Diabetes Risk Estimation." International Journal of Computer Applications 73.10 (2013).
\item Barber, Shaun R., et al. "Risk assessment tools for detecting those with pre-diabetes: a systematic review." Diabetes research and clinical practice 105.1 (2014): 1-13.
\item https://en.wikipedia.org/wiki/Diabetes\_mellitus
\item http://www.webmd.com/diabetes/guide/diabetes-basics
\item https://my.clevelandclinic.org/health/articles/diabetes-mellitus-an-overview
\item https://www.cdc.gov/nchs/nhanes/about\_nhanes.htm
\item https://github.com/SilentSpringInstitute/RNHANES/blob/master/vignettes/introduction.Rmd
\end{enumerate}


\appendix
\section*{Appendix A.}
Some more details about those methods, so we can actually reproduce them.

\end{document}
